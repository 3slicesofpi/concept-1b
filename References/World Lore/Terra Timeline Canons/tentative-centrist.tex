% <!-- todo: prettify this mess. -->
Authormade.
This timeline meets the Feb 2024 requirements as found in TerraTimelineCanons.README. This timeline is accepted as canon. 

Introduction:
    This is a generic timeline. 2022 to 2045, inclusive of Final War Lore.

Version: public-alpha-stable2
    This document has been written and formatted for public display and use.
    The contents of this document is in an early stage of development. All changes are not final and contributions are welcome.
    The contents of the document are internally consistent and mostly do not contradict each other.
    Document is formatted in .txt though markdown artifacts (comments) are present.

Plotpoints/Events:
    The fall of the Unipolar World Order
        USA falls from grace and loses its position as the premiere world power on the global stage. China, various European nations (EU) and even Iran will take over USA's place.
    The fall of the Rules-based World Order
        Nations can use a variety of methods to manipulate their opponent's population to act on their behalf. With such powerful tools of covert influence, even regional powers like Iran can become above the law.
    World War Three
        China and the USA eventually clashing. 
        The Second Cold War, with all the proxy battles are simply a lead-up to all of this. Yes, twenty years was indeed needed to ignite a war between to global powers.
    Singapore getting Nuked
        An Ohio-Class SSBN, despite being wounded, still presents a credible threat to ANY nuclear force. Hypersonics were used to deliver the deadly nuclear payload.
    Putin Getting What He Deserves (Optional)
        Just a personal one from the author. What's there to love about an ex-KGB agent and his oligarch cronies?

Timeline/Periods
    Russian Invasion of Ukraine.
        Justification:
            None needed, real-life event.
        Actions:
            The Russian federation launches an all-out invasion of Ukraine.
            Despite initial setbacks, the Ukrainians successfully manage to push the Russians back and achieve a stalemate.
            The frontline will not shift for years despite extensive efforts from both sides to break the stalemate and regain initiative.
        Effects:
            Russia's invasion was viewed as an imperialist act by other nations. Hence the Russian Federation was struck by international sanctions which impacted the Russian economy, the ruble and many governmrnt mechanisms. Only the quick and decisive actions of treasury officials stalled the Ruble's downward spiral, preventing a major recession.
            Russia devotes a large amount of resources to fighting in the war. Most of it pre-war ammunition and reserve equipment are spent, forcing Russia to enter a war economy state and mass-produce things for the military. This has allowed Russia to maintain the initiative on the battleground. However, pouring money and state resources into the arms industry has left the civilian sectors floundering.
            Russia will look towards its allies for assistance. Iran offers large amounts of kamikaze drones used to great effect in Ukraine. It is also theorized that Iran influenced its proxy, Hamas, to invade Israel to draw Western attention away from Ukraine.
            North Korea trades technology for vast amounts of ammunition. Despite widespread complaints that Korean shells are of lackluster quality, they are still instrumental in keeping Russian stockpiles filled and maintaining artillery fire superiority.
            China offers to trade with Russia and even offers to help bypass sanctions by acting as an intermediary. Knowing that Russia has no one else to trade its vast energy and agriculture exports to, China demands deep discounts and steep markups from its ally.
            Thanks to aid from the West, Ukraine would be able to defend its territory from the numerically superior West and sustain its floundering economy.

    Gaza Crisis
        Justification:
            None needed, real-life event.
        Actions:
            The Hamas militants attack Israeli-controlled territory, taking hostages and killing civilians. In response, the Israelis strike back with overwhelming force, deploying troops into the Gaza Strip and enforcing a blockade.
        Effects:
            Israel's aggressive actions have led to widespread civilian casualties, destruction of property and civil unrest. Thus Israel's actions have seen heavy opposition from around the world with its longstanding ally the US even threatening to cut aid in the first months of the event. In the face of such pushback from the global community, Israel eventually accepts a ceasefire soon after the conflict begins.
            Spurred on by Hamas' decisive actions and the lack of pushback from the global community, other militant groups around the Middle East come out of the woodwork. The Houthis, supported by regional power Iran, hog the spotlight as they hurl missiles at global shipping, provoking a military response from the US.
    
    Russian Disinfomation Campaign
        Justification:
            Russia finally understands that overt actions such as invading a neighbour and cutting gas supplies are unlikely to benefit itself.
            Russia instead increasingly relies on its international propaganda apparatus to discredit and weaken its opponent from the inside, using their own people against themselves.
        Actions:
            Russia leverages its army of bots and trolls, a sophisticated network of influencers and agents as well as allies all over the world to thoroughly saturate the internet with propaganda. Constant reminders of the West's previous follies fill social media while forums and blogs become indunated with praise for Putin. All over Youtube and Reddit, talking heads curse and swear at the incumbents, pointing at minor errors and mistakes to justify their shaky narratives while espousing the opposition for supporting vague and backward ideals. In the halls of the European Union, Russian supporters stall and veto every piece of legislature, slowing bureaucracy to a grinding halt. These varied and far reaching attacks successfully manipulate the US elections and sabotage the EU from within.
            Division and strife now plague the West.
        Effects:
            The West, particularly the US and EU, the primary rivals of Russia are severely weakened and divided. Its citizens quarrel and riot while their leaders struggle to pass policies without erupting into fights.
            Some security agencies in Western countries were alarmed by Russian manipulations and devise their own countermeasures.

    Isolationism and Amerixit (2024, right after the elections conclude.)
        Justification:
            For the 2024 Elections, Russian propaganda boosts a presidential candidate with some radical ideas. He calls for increased federal control over corporations actions outside the US as well as a policy of deregulation within the country. He also claims to oppose the 'globalists' influence and vows to bring jobs back to the US through a policy of isolationism.
            His straightforward promises hooked Americans all over the country and, with some nudging from the internet, the people are convinced.
            On the other hand, the incumbent struggles to find support. Despite offering more than his opponent, a lack of viewership and constant attacks from internet trolls and discreditors weaken the incumbent's stand.
        Action:
            The radical candidate wins by a large margin, not only seizing the presidency, but also securing the most of the crucial House and Senate. This allows laws to be passed with ease.
            Without much pushback, a series of policies were passed in rapid succession to reduce foreign dependence. Corparations were forced to reshore their assets and manufacturing was moved back. Tariffs were raised and protectionist laws were imposed. With the foreigners barred, millions of new jobs were created.
        Effects:
            The negative effects of the policy of isolationism are:
                Catastrophic Disruption of Supply Chain
                Without easy access to the global market, many corparations are unable to source components for their products, resulting in factory stoppages and goods shortages. Though this problem would be mitigated when new factories opened in the US, supply chain disruptions impacted the economy severely, reducing production.
                Impact on Investments
                Foreign direct investment (FDI) decreases as international investors become hesitant to commit capital in an environment of economic uncertainty. This could resulted in a reduction of funding for projects, innovation, and job creation within the US, causing economic stagnation.
                Foreign Competition
                Though companies managed to adapt to the policy of isolationism by bringing back all their industry and workers to America, the cost of doing this was heavily felt. This meant that many companies had to increase the price of their products, making them less competitive than their foreign counterparts.
            This caused the US economy to enter a deep recession. Think Brexit but stupider.
            Around the globe, US decoupling from the global economy sent supply chains into turmoil, causing severe economic problems for many nations as their exports crumbled and US companies left in droves. China was also affected by the US policy of isolationism as it lost a crucial export partner and investment. 
% <!-- this is stupid, this is really really stupid. We all know that mncs will just use shell companies to bypass the tariffs (or some bullcrap method), but this makes more sense somehow. replace it the moment it makes the brain hurt more than it does now. -->    
            Without trade, the US lost much of its influence around the world. Countries in Africa and South America left the US sphere of influence completely to join the Chinese bloc, South East Asian nations drifted away while the Europeans became more independent. China and even Iran capitalized on this by going on an international exposition campaign, making new friends, gaining new allies and bringing more nations into their sphere of influence.
            Military aid to various US allies were reduced or cut completely. Ukraine and Israel stopped recieving weapons and money completely while the US military withdrew forces from East Asia and Saudi Arabia.
            Though Ukraine managed to cope with the lack of aid with a combination of European support and smart internal policy, Saudi Arabia struggles to mainstain control over its more problematic areas as its bloated and corrupt US-trained military is unable to combat rebel militants.
            Japan and South Korea respond to the US withdrawal with an increase in military spending to bolster their forces against increasingly daring Chinese forces. The loss of US influence in Taiwan causes it to drift completely into Chinese influence.

    Reunification (late 2027)
        Justification
            When US political and economic influence in Taiwan disappeared due to US policy of isolationism, the Chinese were able to gain the support of the Taiwanese people.
        Action
            The Chinese government offered a good reunification deal to the Taiwanese. Without US influence, the politicans found little pushback when campaigning for reunification.
            In a historical moment, the Taiwan people vote to join the People's Republic in a democratic referendum as a special territory, finally completing the Chinese dream of reunification.
        Effect
            With this success, China is spurred on to expand its influence not only in East and South East Asia, but also in Russia's own backyard, Central Asia.
            Economic investment into Myanmar and Laos ailing economies, which was previously hit hard due to the US policy of isolationism earns China their friendship.
            Like dominoes, nations previously under US influence enter China's sphere of influence instead, spurred in by new investment. 

    The DC Massacre (2026-2027)
        Justification
            The radical president's policy of isolationism brought the US economy to its knees, causing inflation to skyrocket, standards of living to plummet, severe goods shortages and a major recession.
            Even with propaganda outlets spewing propaganda all over the internet, it was clear to everyone that the current president was unsuitable to lead the country. Even in 2025, even though the policy of isolation had yet to take effect (everyone was given one year to adapt to the new policy.) there were already growing calls for his impeachment.
        Action
            An enraged mob flooded Washington DC, filled the streets with flags and umbrellas (it was raining) and occupied the White House and Capitol building, looting and burning as they went. Another large group consisting of hardcore loyalists and counter-protesters appeared shortly after, clashing with the mob in a massive street fight, laying waste to the city as they fought. Leveraging on the uncoordinated nature of the mob, the loyalists could force their way to the gates of the White House despite the mob's bigger numbers. However, the group there was far more coordinated than the mob in the streets and could keep them out of the building.
            During the battle, the fire prevention system was heavily damaged and/or sabotaged which allowed a deadly blaze to spread all over the building. The people in the building quickly noticed the fire and rushed to escape but found the main entrances blocked by their own barricades and a mob of angry loyalists. With a raging fire on one side and a raging mob on the other, the occupiers were trapped in the burning building and slowly burnt to death.
            When the people heard that the loyalists had trapped a group of their fellow men in a burning building to die, they became frenzied. They brought out handguns and rifles and occupied the cities, shooting everyone suspected to be a loyalist. In Washington, the rioters got their hands on automatic firearms and battled in the streets, turning the city into a warzone.
            Over the course of the next several days, the US military had to be sent into the war-torn cities of America to put down the lynch mobs and control the angry population, slowly restoring peace and order.
            As a result of this event the current president was impeached and a new President emerges from the rubble.
        Effects
            The new President immediately declares martial law and has the military take control of the cities to prevent further fighting. A curfew is declared and a nationwide campaign of arrests are made. Though a few armed groups try to resist, they are quickly located and ruthlessly eliminated.

    An American Miracle (2027-2032)
% <!-- not satisfied with this but will work for now. -->
        Actions
            The new president implements a new five-year-plan to bring the US economy out from its state of turmoil. He also reverses many of the isolationist policies pushed by his predecessor. He also goes on an international foreign relations campaign to restablish ties and bring back foreign investment into America. Finally, he passes laws to regulate the internet and reduce divisive conflict as well as creating a new government agency to counter Russian propaganda and extremist disinformation.
        Effects
            Miraculously, the US economy manages to rebound, returning to its original state by 2032.
            The new government agency soon uses their newfound power to spread American ideals all over the internet by way of trolls and botnets similar to the Russian propaganda apparatus.
        
    Middle East Chaos (2026-)
% <!-- not much happens here so no need for jae -->
        In the Middle East, The veneer of peace that defined the prior era has been stripped away. Now, militants and rebels take to war, fighting fiercely in a deadly struggle for dominance. Government forces and dictatorial cults fight in a hopeless battle to maintain the status quo while rebels and insurrectionists steadily grow in power and strength as they strive for legitamacy and strength. One by one, the nations of the MIddle East find themselves embroiled in civil war. Syria and Iraq, as always, have been in a state of anarchy while the Taliban's control in Afghanistan has collapsed fully. The giants of the Middle East, Saudi Arabia and Iran are struggling to fend off each others proxies in Yemen and Kurdistan respectively.
    
    The Sick Man of Europe (2029-2031)
        State of Russia:
            Struck by disease and the ravages of age, Putin retreats from his office to nurse his illnesses. With their leader's death within sight, high ranking officials, influential oligarchs and mercenary leaders make preparations to take over the nation upon Putin's death. Some find support with China and Iran while others become proxies of the Europeans.
            The war has also taken its toll on the Russian economy. When the factories switched to producing military materiel, less consumer goods were produced, causing a decrease in food supply. Many factories also relied on Western components for their machines, which were hard to substitute. Though extensive efforts were made by the government to indigenise local industries, critical machines and infrastructure began breaking down from lack of maintainance, causing stoppages in factories. Coupled with the government diverting replacement parts to military producers instead of civilian factories, supply of essential goods decreased an prices skyrocketed.
            Though Russia was able to get goods and substitute machine parts from its ally China, it was often sold at a huge markup by companies looking to make a quick buck. Local regions seeking better prices began independently negotiating trades with other powers such as Iran, China and even the Europeans. The Western powers also approached disgruntled governors, ambitious oligarchs and opposition leaders in hiding, pulling them from the Putin's influence.
    
    Warlord Era (2032-)
        Justifications
            see The Sick Man of Europe
        Actions
            Sensing weakness within Russia, the Ukrainians launch a daring raid into Russia in the middle of winter, capturing a large city and several towns with little resistance. Putin, still bedridden from his illness, orders Russian nuclear forces to high alert and demands the Ukrainians withdraw immediately.
            As per protocol, upon reception of the presidential order nuclear siloes were armed, rocket forces mobilized and dispersed, and all submarines capable of battle sent out to the sea. All eyes are on Russia and Ukraine. Thousands of military personnel have also been withdrawn from the frontline and sent back home to defend the country proper.
            Only a month later, Vladimir Putin dies from excessive stress, quickly being replaced by an acting government. However, the temporary administration is unpopular among the oligarchs and its members are quickly replaced by their selections. However, things turn ugly quickly when the oligarchs disagree amongst themselves.
            Soon, disagreements turn into skirmishes, which turn into full-blown war, complete with mercenaries fighting in the streets. The Russian Army is recalled from the front to restore order, but it is revealed that their commanders were controlled by the oligarchs. Civil war erupts in Russia, with each side fighting for their own selfish desires.
        Effects
            With the breakdown of order, the looting and burning begun. Many convict-turned-conscripts in the Russian Army escaped in the confusion of redeploying from the frontlines. Some of them stole from unguarded homes and stores while others set fire to buildings. Cults and gangs were free to rear their ugly heads, taking over abandoned army bases and stealing the weapons inside, which they use to carve territory for themselves. 
            Eventually, lines of control start to form between the oligarch-warlords' mercenary forces. As each side consolidates their gains, it becomes harder to take territory without suffering heavy casualties. As each sides thinks of ways to overpower the other, the eyes finally turn not to the bloodied streets of Moscow, but to the armed nuclear warheads dispersed all over Russia. With each one capable of leveling cities and destroying armies, everyone realises the importance of obtaining one.
            Thanks to Putin's order, the Russian nuclear forces were dispersed all over the country and at high readiness. When the country fell to anarchy, their operators did dismantle and sabotage the delicate warheads in an effort to prevent them from falling into the wrong hands. However, the sabotaged warheads could easily be repurposed into a 'dirty bomb' capable of irradiating cities. 
    
    Special Military Operation into Russia (autumn 2032)
        Justification
            Amid the flood of refugees entering Europe and concerns about live Russian nuclear weapons, the UN finally sends a task force to stabilize the situation and recover the nuclear devices.
            Maintain Nuclear non-proliferation at all cost.
        Action
            A multinational task force of Chinese, European, Central Asian and US forces entered Russia and eliminated the warlords and their mercenary forces. The pacification and stabilization of the Western and Far East regions proceed relatively smoothly, thanks in part due to already-present allies there and proximity to allied countries.
            However, when the UN forces pushed deeper into Russia, a 'dirty bomb' was detonated as a warning to the UN forces not to advance any further. Knowing that the Russian warlord oligarchs now had access to Russian nuclear weapons, the UN had no choice but to retreat and avoid provoking the nuclear-armed states.
        Effects
            The Russian warlords have inherited much of the Russian Federation's conventional arms and a sizable fraction of its nuclear arsenal. They sit in the middle of the oil and resource rich Siberia, which is easily defended due to its challenging weather and difficult terrain. However, the warlords cannot fully harness this capablility as they are still divided and their economies are small.
            The warlords soon align themselves with other nations such as China and the Europeans to bolster their weakened economy and reenter the global stage as a sovereign nation, eventually ceding their nukes for a peaceful existance (in 2040.)
            Other warlords chose a more adventurous option instead, aligning with rogue nations like Iran and the islamic militants in the Middle East. In exchange for a supply of Russian arms, armour and advanced technology, the warlord states top leadership can enjoy a life of luxury provided by Iran as well as a guarantee of a safe haven if anything goes wrong.
            With this arrangement, Russian arms flow freely from the frigid north to the arid deserts of the Middle East, greatly strengthening the militants there while Iran enriches itself with valuable Russian technology. It is also rumored that nuclear weapons were moved to the Middle East, which would spell disaster if true. Imagine a nuclear-armed ISIS!
            The Special Military Operation was expected by many observers to bring the powers of the world together, mending grudges and making bonds, to prove to the world that the US and China had settled their differences and were ready to coexist. However, the opposite happened.
            Initially, the US demanded to lead the operation and have the multinational task force under its command. China disputed this, stating that indepedence would be more beneficial to the operation. Both sides eventually relented and entereed the operation under three seperate command groups, one under the US near Europe, an East Asian front led by China and a smaller Central Asian command group with Kazakhstan at the helm. This arrangement did not come by without much quarreling between the two powers and significant delays, allowing the Russians to stabilize and consolidate their positions.
            The UN task force entered the operation with the naive thought that it would be a quick and easy operation; simply show them the might of the united international community and liaise with the shocked oligarchs to secure peace. This sentiment did not persist after the first few days however when the population fought back bitterly. Cracks began to form between the various militaries as they struggled to work together to pacify the region. The US military was frustrated by Chinese refusal to provide intelligence and the Chinese were likewise angry about the sluggishness of US air support, which they claim had indirectly caused the death of many of their servicemen.
            A dilipated MiG, a relic from the Cold War lifted off from a cratered Russian highway. Under its belly held a large and onimous object, a nuclear 'dirty bomb'! Chinese intelligence assets were able to track the bomb's movement from a deep underground base to the makeshift airstrip and to the sky and thus warned their own troops of the incoming warhead. However, this vital information did not make its way to the Americans early enough, which caused the deaths of hundreds of US and UN military servicemen when the bomb struck their position. When the US heard that China had intelligence that the nuke was coming towards them but did not issue a warning, everyone assumed the worst. This marked the end of US-Chinese cooperation in the operation. The special military operation in Russia would end soon after to avoid provoking the new nuclear-armed powers.
    
    Second Cold War (2033-2044)
        Justification
            Though both USA and China did work together to secure the Russian territories, misunderstandings between the two powers during the operation caused further division.
            The Chinese and the US had both created vast propaganda apparatuses that regularly flooded the internet with propaganda and disinformation. Leveraging an army of paid internet influencers, banks of botnet servers and teams of professional hackers, each side fought relentlessly to take control of the airwaves. Subjecting the common people to huge amounts of propaganda, both sides managed to influence the people from the opposing nation to support themselves instead but had to endure waves of indoctrinated citizens back home.
        State of the War
            Tensions between USA and China is high. The Chinese high sea fleets stare down US carrier groups in the South China Sea and off Taiwan. Both sides also support proxies to gain an advantage over each other. These proxies operate in the warlord regions in Russia (mostly as rebel forces), parts of Central Asia and all around the Middle East, using  weapons supplied by their backers to fight each other in a battle for dominance.
% <!-- insert something about internet discourse here -->
    
    The Final War / WW3 (2044-2045)
        Justification
            Even before tensions between China and the US had appeared, the two Koreas were locked in a bitter rivalry. Having fought a tumultuous war for survival in the Korean War, each side built up massive militaries in anticipation for another conflict. Now, as the cold war between their benefactors, China and the US, heated up, both sides seem intent on locking horns, sparking a new war on the peninsula.
            Escalation came slowly and gradually. However, due to the prevailing attitude in both their countries, neither was keen to descalate. Moreover, both the US and China were much more occupied with the chaos in the Middle East and the Russian Warlord Regions and so intervention was rare.
            By 2040, each side was at each other's throat, itching for a reason to fight. There had already been countless examples of border intrusions, military raids and covert operations by both nations and their respective populations clamoured for retribution. However, wiser minds held out despite all the chaos, thus keeping the peace.
            A global insurgency cult, funded by profits from the arms trade in the Middle East, manages to get its hands on a nuclear device. By disassembling it into small components, they are able to smuggle it into South Korea, where it is secretly reassembled as a 'dirty bomb'. The bomb used igniters and explosives scavenged from artillery shells to compress a small mass of enriched uranium to supercriticality, creating a small nuclear explosion. However, the bomb's lethality lay in the tonnes of radioactive matter arranged around the jerry-rigged nuke.
            A mushroom cloud erupts over northern Seoul, burning thousands in a fiery blast. The explosion sent radioactive material into the air, where it spread all over the country. Over the next few days, a million more would be consigned to death and disease due to radiation poisoning while millions more would be displaced, forced to leave their homes.
            The South Korean military reacted immediately. Interpreting the attack as a North Korean first strike, they eliminated their military installations and missile siloes within minutes of the nuclear blast. The North Korean leadership, unaware of the incident in Seoul, responded with an all-out attack of their own.
            After a short struggle near Seoul, the North Koreans were beaten back and had to retreat to the Chinese border, with the defeat of the communists imminent. However, this did not sit well with the Chinese as sharing a border with a US ally was highly disadvantagous. However, China had prepared for this eventuality and already moved several units to help their ally. Reinforced by the powerful Peoples' Liberation Army, the North Koreans were able to conduct a successful counterattack. By this time, US reinforcements had arrived, preventing the South from falling to the communists.
        Action:
            The Second Korean War inevitably spilled out of the peninsula, triggering war between the US and China. The North Koreans first used Chinese-supplied anti-ship missiles to target US reinforcements traveling by ship. The US retaliated with strikes against Chinese supply convoys and arms depots, even against those within the Chinese mainland itself. The Chinese viewed this as a sign that the US was willing to, and capable of, escalating the conflict. To secure victory before things got worse, China deployed more troops, as well as modern SAM systems and stealth fighters to negate the US advantage in airpower. For the US, the sudden appearance of advanced Chinese air-denial assets resulted in the destruction of many of their own planes, which had only fought old Soviet planes and carried bombs instead of missiles since the start of the war. Strikes against air defense sites and airfields all over China and Korea were conducted, triggering yet another escalation in the war.
            Both sides fought long and hard, slowly introducing new capabilities as the war escalated. They also expanded the war to include parts of the Middle East and fought in limited skirmishes in Central Siberia whilst bringing out novel warfare techniques like hacking and economic sabotage to gain the upper hand. 
            However, due to US advantages in air power, China found itself gradually losing ground in Korea and had already lost the city of Pyongyang to the South. Additionally, the retasking of two new US Navy Supercarriers to the Korean conflict to assist the two already there would shift the balance of power to favour the West. 
            However, China had already set up a cunning plan to destroy US naval assets, using a combination of intelligence gathered from covert operations like hacking and spying, constant tracking via satellites and the use of an overwhelming number of ordanance. This strike would totally wipe out all of US naval capabilities in the area before the Western forces could respond and they would be totally helpless with Chinese naval and aerial superiority secured. Though both sides had fought with one hand tied behind their back since the start of the war, Chinese leadership hoped that this daring attack could cut off the enemy's limbs.
            The positions of the four major US Navy fleets were being carefully tracked using a sophisticated intelligence network of advanced satellites and stealth drones. Two of the four fleets, as well as a significant number of US Navy and Marines ships were gathered around the Korean Peninsula, mercilessly bombarding the Chinese and Korean forces. Another two fleets, each one with a valuable and powerful nuclear-powered Ford-class supercarrier at its core, were currently passing through the East China Sea.
            At the same time, extensive efforts were made to keep the plan secret. Several rocket units were moved into position under the guise of a routine training exercise. Hundreds more were carefully positioned, one by one in dispersed and hidden positions. All this movement was obfuscated by a mass mobilization order which served well to distract Western observers. 
            As the West ridiculed China for enacting a deeply unpopular mass mobilization order, Chinese rocket forces moved out right under their noses, setting up in secret firing positions all over the coast. As the Chinese people rose up to protest the mobilization order, the launchers pointed their missiles to the night sky while missile-laden strike fighters cruised eastward. As the weary radar operators on the bridge sat inattentively at their consoles, they noticed an large anomalous blip appear on their screens.
            The operators stared at their consoles in disbelief, but the blip kept getting larger and larger. New blips also appeared all over the screen, over the coasts of China and Korea, over identified Chinese naval vessels and even over the empty sea. As the blips moved closer and closer to the US fleets, it became clear to everyone that the Chinese had launched an all-out attack.
            Everything was being hurled at the Americans. Hundreds of sea-skimming cruise missiles of every variety barreled towards their targets while giant rockets, both of Chinese and Russian origin, travelled along their ballistic arcs through the upper atmosphere, ready to release a rain of warheads on their enemy below. The most advanced hypersonics made their debut here, charging at five times the speed of sound, eager to prove their worth while the veritable 'Onyx' leaped out of their launch tubes from hidden Russian submarines, their age not impeding their primary function. All these missiles were launched in massive, carefully timed salvoes such that every piece of ordanance would strike their target at the same time, overwhelming defenses.
            Now for the defense. Massive radar equipment spooled up, shining powerful beams of radiation at the night sky, locating the barrage of projectiles coming their way. The ships' crews, now awakened by the call of the general alarm, rush about to coordinate fire-control systems and test vital equipment. Short-range missile launchers turn and stand ready, targeting radars aboard every ship pivot towards the oncoming projectiles, heavy laser and maser systems charge their capacitators and VLS cells open their lids, exposing the first salvo of interceptors to the cool night sky.
            A mere fourty-five seconds after the first Chinese missiles are detected, long-range interceptor missiles leap out of their cells by the hundred. The ballistic missiles are targeted first as they are easier to locate and the old AEGIS systems are more than up for the task. Older Standard-class missiles lunge at the ballistic missiles and when some miss, more advanced missiles change course and eliminate them instead.
            Even as the long-range missiles were launching from the escorts, the Ford-Class supercarriers had started to scramble their carrier air wings, sending their squadrons of missile-laden stealth fighters to the skies. These were reinforced by more aircraft launched from airbases in Korea and in the Philippines. Confronting this massive force was multiple squadrons of Chinese stealth fighters, ready to hunt down US aircraft with their advanced long-range missiles. A massive aerial brawl thus ensued under the night sky, painting it white and yellow with missile trails, afterburner smoke and explosions. After suffering tremendous casualties, US pilots manage to wrench a few small sections of airspace from the relentless assault of the Chinese, creating areas of regional air superiority. This allows pilots to finally target some of the incoming antiship missiles with their remaining air-to-air missiles.
            By this time, the incoming salvo of Chinese munitions has entered medium-range SAM range and are once again engaged by the escorts. Scores of enemy missiles fall under the combined assault of the fighters and SAMs, thinning their overwhelming numbers. But it is not enough. Of the one thousand or so missiles that had been launched, only a mere two-thirds had been neutralised. The sheer number of incoming firepower was too much, even for America's formidable air defenses. The variety of missiles they had to face also complicated fire control and command, hampering effectiveness.
            Calculating that their fleets' short-range air defenses would be totally incapable of taking down the incoming onslaught, three out of the four fleet commanders chose to abandon their ships, leaving the ships' automated defense system to make a final stand. They fought defiantly for a short moment, but eventually succumbed to the tide of fire. Without damage control teams to repair the crippling damage, each ship in the three fleets sank.
            The fourth fleet of the ill-fated group, the 9th Fleet, was somewhat luckier. Having left port late, it was trailing behind, still well within the Pacific when the attack happened. However, this meant that only longer-range munitions could hit the 9th Fleet from China, forcing the Chinese forces to use old and obsolete ballistic missiles while the air-launched long-range anti-ship missiles were easily shot down by US aircraft. However, the older systems onboard the aging ships of the 9th Fleet struggled to engage so many targets at once, nor were they able to effectively bring down the most advanced hypersonics.
        Effects:
            The 9th Fleet did not come out of the attack unscathed. In fact, eight hypersonic missiles had struck the side of the Ford-class supercarrier, instantly dooming it. A variety of missiles also impacted its escorts at the same time, causing heavy damage. However, nearly a third of the escorts had survived due to effective damage control by the ships' crews.
            Of the 11 guided missile cruisers and destroyers of the 9th Fleet, merely four remained, heavily damaged and scarred but still floating. They carefully manuevered to the friendly port of Singapore, where they could expect fuel and some provisional repairs.
            Under the sea however, another war was happening. It was a battle of patience and wit, a test of silence and defiance. A fight between submarines against everything else on the surface. Attack submarines striked at distracted warships above water while evading anti-submarine warfare assets at the same time. But among the latest stealthy diesel-electric attack submarines and the numerous anti-sub assets scouring the waters, a massive nuclear-powered Ohio-class ballistic missile submarine was like a butterfly among bees.
            Miraculously though, it managed to sneak out of the chaotic East and South China Sea, only suffering a single hit from an anti-submarine torpedo when it charged into Malaysian waters. It then limped to the port of Singapore, where it hoped to hide inside the congested waters of the Straits.
            Before the four fleets were utterly decimated, they managed to release strike packages on the Chinese high seas fleets using a combination of stealthy cruise missiles and air-launched anti-shipping ordanance. While not as devastating as the Chinese saturation attack, it managed to disable nearly half of the Chinese fleet. B-series stealth bombers then proceeded to rain fury onto the most populated Chinese cities, toppling skyscrapers with tonnes of thermite. Soon, both sides decided to launch their nuclear missiles, ending it all.
            An Ohio-class SSBN is a very powerful asset. Holding nearly a hundred nuclear warheads, it could bring any mid-sized country to its knees single-handedly. However, they were old designs and lacked many of the low-observability features of their more newer, advanced attack submarine cousins, hence Chinese anti-submarine assets could easily locate and destroy one of these monstrous monsters. It was only a miracle that ensured that the stranded SSBN managed to get that far without being engaged.
            Just as the SSBN limped to the waters of Singapore, one of the sonar operators aboard a US cruiser found its sonar return among the flotilla of merchant ships stranded in the Straits when he was conducting some simple tests. However, as the console identified the submarine's large and distinctive sonar return as that of a (slightly dented) allied Ohio-class SSBN, he did not choose to engage it. Instead, he told his friends about what he saw on the console.
            The news spread from sailor to sailor, and eventually to the ears of Chinese military intelligence. When they realised that an expensive and powerful strategic asset was hiding in plain sight together with the remnants of the defeated 9th Fleet, a nuclear-capable hypersonic missile would be tasked to eliminate the vulnerable SSBN before it could unleash its own missiles.
            When the nuclear exchange started, the Chinese did not forget to lob a trio of advanced hypersonic nuclear missiles at Singapore. Though the Malaysian and Singaporean militaries boasted highly capable systems, they were both surprised that they had been targeted with such overwhelming force and were not fully ready. Regardless, they managed to bring down two of the city-killers. 
            As the last nuke decended upon the city, a blinding light suddenly appeared and rose up from the city skyline; it was a tiny red ball, shining with incredible luminesence,appearing to challenge the apocalyptic weapon of destruction coming its way. It collided with the nuke head-on, swallowing the warhead whole.
            The nuclear warhead detonated, releasing a massive burst of energy and radiation. This was promptly absorbed by the red ball, which would go on to turn into a dome and expand rapidly to cover the entire city, emitting a dazzling white light, blinding everyone and everything within as it grew.
            When the lights finally dimmed, the nation of Singapore found itself beneath a quiet, star-lit sky, with only calm and peaceful waves and the gentle caress the wind.